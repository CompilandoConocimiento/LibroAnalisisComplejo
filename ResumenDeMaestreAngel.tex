\documentclass[11pt,letterpaper]{article}

\usepackage{graphicx}		%Paquete para anexar fotos
\usepackage[utf8]{inputenc}
\usepackage[english,spanish]{babel}
\usepackage{amsmath, amsfonts, amsthm, amssymb, physics}  %Paquetes matematicos
\usepackage{mathrsfs}
\usepackage[margin=2cm]{geometry}	
\usepackage{fancyhdr} %encabezados y pie de pagina
\usepackage{multicol}  %columnas
\usepackage{algpseudocode}	%Los 2 paquetes son para poner pseudocodigo 
\usepackage{algorithm}
\usepackage{listingsutf8}	%	Los dos paquetes son para citar codigo	
\usepackage{xcolor}

%\usepackage[doublespacing]{setspace}	%doble espaciado por linea

%\usepackage[onehalfspacing]{setspace}		%	espaciado por linea de 1.5
%	\documentclass[twocolumn]{report}		%	pone dos columnas

%	======================================================================
%	Permite poner pie de pagina
	\pagestyle{fancy}
	\fancyhf{}
	\rhead{�ngel L�pez Manr�quez}
	\lhead{Matematicas avanzadas para la ingenieria}
	\rfoot{Pagina \thepage}

%	======================================================================
%	Permite el anexo de teoremas, corolarios y lemas
	\newtheorem{theorem}{Teorema}[section]
	\newtheorem{corollary}{Corollary}[theorem]
	\newtheorem{lemma}[theorem]{Lema}
	
	\theoremstyle{definition}
	\newtheorem{definition}{Definici�n}[section]

%	======================================================================
%	Portada
\begin{document}
\begin{titlepage}
	\centering
%	\includegraphics[width=0.15\textwidth]{ipn.png}\par\vspace{1cm}
	{\scshape\LARGE Escuela Superior de C�mputo \par}
	\vspace{1cm}
	{\scshape\Large Notas \par}
	\vspace{1.5cm}
	{\huge\bfseries Matematicas avanzadas para la ingenieria \par}
	\vspace{2cm}
	{\Large\itshape por: \\ L�pez Manr�quez �ngel  \par}
	\vfill
	profesores\par
	Dennis G. Zill \\
	Jean-Baptiste Joseph Fourier

	\vfill

% Bottom of the page
	{\large \today\par}
\end{titlepage}

%	======================================================================

\section{Integracion en el plano complejo}
	
	\theoremstyle{definition} 
 	\begin{definition}{La integral compleja} 
 	La integral compleja $f$ sobre $C$ es

 		\[\displaystyle{ \int_C f(z)dz = \lim_{||P|| \rightarrow 0} \sum_{k=1}^n} f(z_k^*)\Delta z_k \]	
 	\end{definition}

	\begin{theorem}[Evaluacion de una integral de contorno] 
		Si $f$ es continua sobre una curva $C$ dada por la parametrizacion $z(t)=x(t)+iy(t)$, $a \leq t \leq b$, entonces
			\[ \int_C f(z)dz = \int_a^b f(z(t))\dot z(t)dt \]
	\end{theorem}

	\begin{theorem}[Teorema del acotamiento]
		Si $f$ es continua sobre una curva suave $C$ y $|f(z)| \leq M \ \forall z$ en $C$, entonces 
			\[ \abs{\int_C f(z)dz} \leq ML \]
		donde $L$ es la longitud de $C$.
	\end{theorem}

	\begin{theorem}[Teorema de Cauchy-Goursat]
		Suponga que $f$ es analitica en un dominio simplemente conexo $D$. Entonces para cada contorno simplemente cerrado $C$ en $D$
			\[\displaystyle{\oint_C f(z)dz = 0} \]

	\end{theorem}

	\begin{theorem}[Singularidad]
		Sea $C$ un contorno simplemente cerrado que contiene a un solo punto 	singular $z_0$, entonces $\forall n \in \mathbb{Z}$

			\[\displaystyle{\oint_C \frac{dz}{(z-z_0)^n} 
				=	\left\{ \begin{array}{lcc}
             2\pi i &   si  & n = 1 \\
             \\ 0 &  si & n \neq 1
             \end{array}
   				\right.} \]
	\end{theorem}

	\begin{theorem}[Teorema de Cauchy-Goursat para dominios simplemente conexos]
		Sean $C,\ C_1,\ C_2,\ ...,\ C_n$ contornos simplemente cerrados con una orientacion positiva tales que $C_k,\ k = 1, 2, ... n$ estan contenidos en $C$ donde $C_i,\ C_j$ con $i \neq j$ no tienen puntos en comun y $f$ es analitica en el interior de $C$ y fuera de $C_k,\ k = 1, 2, ... n$. Entonces
			\[ \oint_C f(z)dz = \sum_{k=1}^n \oint_{C_k} f(z)dz \]
	\end{theorem}
	 
	\begin{theorem}[Analiticidad implica indepencia de la trayectoria]
		Suponga que D es un dominio simplemente conexo y C una trayectoria en D, entonces $\displaystyle{\int_C f(z)dz}$ es independiente de la trayectoria. 
	\end{theorem}

	\begin{theorem}[Teorema fundamental de las integrales de contorno]
		Sea f una funcion continua en D, sea F la antiderivada de f en D, Sea C una camino con un punto inicial $z_0$ y un punto final $z_1$, entonces 
			\[ \int_C f(z)dz = F(z_1) - F(z_0) \] 
	\end{theorem}

	\begin{theorem}[Existencia de la antiderivada]
		Suponga que f es analitica en un dominio simplemente conexo D, entonces $\exists F$ para f en D.
	\end{theorem}

 \begin{theorem}[Formula de la integral de Cauchy]
		Suponga que f es analitica en un dominio simplemente conexo D y C una curva simplemente cerrada, donde cada punto de C esta en D, entonces para un punto $z_0$ encerrado por C

			\[ f(z_0) = \frac{1}{2\pi i} \oint_C \frac{f(z)dz}{z-z_0}	\]
	\end{theorem}

 \begin{theorem}[Formula de la integral de Cauchy para derivadas]
	Suponga que f es analitica en un dominio simplemente conexo D y C una curva simplemente cerrada situada enteramente en D, entonces para un punto $z_0$ dentro de C

		\[ f^{(n)}(z_0) = \frac{n!}{2\pi i} \oint_C \frac{f(z)\ dz}{(z-z_0)^{n+1}}	\]
	\end{theorem}
 
\section{Series y el teorema del residuo}

	\begin{theorem}[Criterio de convergencia de una sucesion]
		una secuencia $\{z_n\}$ converge a un valor L tanto $Re(z_n)$ como $Im(z_n)$ convergen.
	\end{theorem}

	\begin{lemma}[Reordenacion de una serie de potencias]
		Una serie de potencias converge absolutamente dentro de su radio de convergencia. Como consecuencia, se pueden reordenar los sumandos sin el problema de afectar su valor de convergencia.
	\end{lemma}

	\begin{theorem}[Continuidad]
		Una serie de potencias dentro de su radio de convergencia representa una funcion continua.
	\end{theorem}

	\begin{theorem}[Diferenciacion e integracion de una serie de potencias]
		Cualquier serie de potencias es diferenciable en su radio  de convergencia R e integrable bajo una curva simple cerrada C, donde C esta situada en el R.
	\end{theorem}

	\begin{theorem}[El teorema de Taylor]
		Sea f una funcion analitica en un dominio D y $z_0$ un punto en D, entonces f tiene una representacion en series de la forma

			\[f(z)=\sum_{k=0}^\infty \frac{f^{(k)}(z_0)}{k!}(z-z_0)^k\]
		
		Si el circulo de convergencia de la serie esta en D.		
	\end{theorem}

	\begin{theorem}[El teorema de Laurent]
		Suponga f analitica en un dominio D definido por $r \le \abs{z-z_0} \le R$. Asi, f tiene una representacion en D de la forma

			\[f(z)=\sum_{k=-\infty}^\infty a_k(z-z_0)^k\]

		donde los coeficientes $a_k$ estan dados por
			\[ a_k = \frac{1}{2\pi i} \oint_C \frac{f(s)\ ds}{(s-z_0)^{k+1}}, \quad k \in \mathbb Z \]

		con C un contorno simple cerrado en D y $z_0$ esta en su interior. 
	\end{theorem}

\section{Series de Fourier}
	\begin{definition}{Producto interno}
		Se dice que una funcion (,): $V \rightarrow \mathbb K$ es un producto interno si para $u,\ v,\ w\ \in V$, con $V$ un espacio vectorial, $\alpha \in \mathbb K$ si se cumple que

		\begin{multicols}{3}
			\begin{enumerate}
				\item $(u, v) = (v, u)$ 
				\item $(\alpha u, v) = \alpha (u, v)$ 
				\item $(u + v, w) = (u, w) + (v, w)$ 
				\item $(u, u) > 0$ si $u \neq 0$ 
				\item $(u, u) = 0$ si $u = 0$ 
			\end{enumerate}
		\end{multicols}
	
	\end{definition}

	\begin{definition}{La norma de un vector}
		La norma de un vector bajo un producto interno definido de un vector $v \in V$ es $ \norm{v} = \sqrt {(v, v)} $
	\end{definition}


	\begin{definition}{Vector ortogonal}
		Se dice que un vector $u$ es ortogonal a $v$ si $(u, v) = 0$.
	\end{definition} 

	\begin{definition}{Conjunto ortogonal}
		Se dice que un conjunto $\{u_i\}_{i=0}^\infty$ es ortogonal si $(u_i, u_j) = 0 \quad i \neq j$.
	\end{definition}

	\begin{definition}{El producto interno de funciones}
		El producto interno de dos funciones $f$ y $g$ en el intervalo $[a, b]$ es
			\[ \int_a^b f(x)g(x)dx \]
	\end{definition}

	\begin{theorem}[Expansion ortogonal en series]
		Suponga que $ \{ \phi _k(x) \}_{k=0}^\infty $ es un conjunto ortogonal en $[a, b]$, entonces una funcion puede ser escrita como
			\[ f(x) = \sum_{k=0}^\infty \frac{(f, \phi_k)}{\norm{\phi_k}^2}\phi_k\]
	\end{theorem}

	\begin{theorem}[Conjunto ortogonal y funcion peso]
		Un conjunto $ \{ \phi _k(x) \}_{k=0}^\infty $ es ortogonal con respecto a una funcion peso w(x) en $[a, b]$ si
			\[ \int_a^b w(x)\phi_m\phi_n dx = 0, \quad m \neq n\]
	\end{theorem}

	\begin{definition}{Las series de Fourier}
		Una funicion $f$ definida en un un intervalo $(-p, p)$ esta dada por

			\[f(x) = \frac{a_0}{2} + \sum_{n=1}^\infty \left[a_n \cos(\frac{n\pi}{p}x) + b_n \sin(\frac{n\pi}{p}x)\right] \]
		donde
			\[ a_0 = \frac{1}{p} \int_{-p}^{p}f(x)dx, \quad
				 a_n = \frac{1}{p} \int_{-p}^{p}f(x)\cos(\frac{n\pi}{p}x)dx , \quad
				 b_n = \frac{1}{p} \int_{-p}^{p}f(x)\sin(\frac{n\pi}{p}x)dx, \quad
				 \]
	\end{definition}

	\begin{theorem}[Convergencia de las series de Fourier]
		Sea f una funcion continua a trozos. La serie de fourier convergera en los tramos continuos a f y convergera a los puntos discontinuos a la semisuma
			\[ \frac{f(x+) + f(x-)}{2} \]
	\end{theorem}

	\begin{definition}{Paridad de funciones}
		Se dice que una funcion $f$ es par si $f(-x) = f(x)$ (simetrica respecto al eje $y$) e impar si $f(-x) = -f(x)$ (simetrica respecto al origen.
	\end{definition}

	\begin{theorem}{Propiedades de funciones pares e impares}
		\begin{enumerate}
			\item El producto de dos funciones, ya sean ambas pares o impares, es par.
			\item El producto de una funcion par con una impar es impar. 
			\item La suma de dos funciones pares da una funcion par.
			\item La suma de dos funciones impares da una funcion impar.
			\item Si $f$ es par, entonces $\int_{-p}^p f(x)dx = 2\int_{0}^pf(x)dx$
			\item Si $f$ es impar, entonces $\int_{-p}^pf(x)dx = 0$
		\end{enumerate}
	\end{theorem}

	\begin{theorem}[Serie de cosenos]
		La series de Fourier para una funcion par en (-p, p) es

			\[f(x) = \frac{a_0}{2} + \sum_{n=1}^\infty a_n \cos(\frac{n\pi}{p}x) \]
		donde
			\[ a_0 = \frac{2}{p} \int_{0}^{p}f(x)dx, \quad
				 a_n = \frac{2}{p} \int_{0}^{p}f(x)\cos(\frac{n\pi}{p}x)dx
				 \]
	\end{theorem}

	\begin{theorem}[Serie de senos]
		La series de Fourier para una funcion impar en (-p, p) es

			\[f(x) = \sum_{n=1}^\infty b_n \sin(\frac{n\pi}{p}x) \]
		donde
			\[ b_n = \frac{2}{p} \int_{0}^{p}f(x)\sin(\frac{n\pi}{p}x)dx	\]	
	\end{theorem}

	\begin{definition}{Serie compleja de Fourier}
		Suponga $f$ continua a trozos en un intervalo $(-p,p)$, entonces 
			\[f(x) = \sum_{n=-\infty}^\infty c_n e^{\pi nxi/p} \]	
			donde 
			\[c_n = \frac{1}{2p} \int_{-p}^{p}f(x)e^{-\pi nxi/p}dx\]
	\end{definition}

	\begin{definition}{Frecuencia angular fundamental y espectro de frecuencia}
		$\omega = 2 \pi / T = \pi / p$ es la frecuencia angular de la serie de Fourier.
		Los espectros de frecuencia son es la grafica $(nw, \abs{c_n})$.
	\end{definition}

\section{La transformada integral de Fourier}
	
	\begin{definition}{Las transformadas de integrales}
		La transformada de una integral es una transformacion lineal $T:t \rightarrow s$ para una funcion $f$ tal que 
			\[ F(s) = \int_a^bf(t)k(t,s)dt \]
		donde $k(t,s)$ es el kernel integral y $F(s)$ es la transformada de la funcion $f$.
	\end{definition}

	\begin{definition}
		La integral de Fourier para una funcion $f$ definida en $(-\infty, \infty)$ es
			\[f(x) = \frac{1}{\pi} \int_{-\infty}^{\infty} A(\alpha) \cos(\alpha x) + B(\alpha) \sin(\alpha x) dx \]
		donde 
			\[ A(\alpha) = \int_{\infty}^{\infty} f(x) \cos (\alpha x) dx, \quad B(\alpha) = \int_{\infty}^{\infty} f(x) \sin (\alpha x) dx  \]

	\end{definition}
	
	\begin{definition}
		La transformada de Fourier se define como
			\[ F(w) = \mathscr F \{f(t)\} =  \int_{-\infty}^{\infty} f(t)e^{-iwt}dt \]
	\end{definition}

	


	\[ x = \left(\sum_{n=0}^\infty  \frac{(-1)^n}{n+1} \right)^{-1}(ln \abs{?}+i\ arg(?)) - 1 \]

	Demostracion

	Haciendo $y = ?$ tenemos: 
			\begin{align}  %  Si se le pone * no se enumeran las ecuaciones
			  2^x + 2^x &= 2^{x+1} = y = e^{(x+1)\ln 2}   \\
			  e^{x+1} &= y ^ {1/\ln 2}	\\
			  x + 1 &= \ln(y ^ {1/\ln 2}) = (1/\ln 2) \ln y \\
			  x &= (1/\ln 2) \ln y - 1 
			\end{align}
	Usando la serie de Taylor para $\ln 2 = \sum_{n=0}^\infty  \frac{(-1)^n}{n+1}$  y considerando $\ln z = \ln|z| + i\ arg(z)$ con $z \in \mathbb C$ para considerar el campo de los complejos, la demostracion queda completada. \qed

	


\section{Apendice}
	\subsection*{Producto a suma}

		\begin{multicols}{2}
			\begin{align*}  %  Si se le pone * no se enumeran las ecuaciones
			  \sin x \sin y &= \frac{1}{2}\big[\cos(x - y) - \cos(x + y)\big]\\
			  \cos x \cos y &= \frac{1}{2}\big[\cos(x - y) + \cos(x + y)\big]\\
			  \sin x \cos y &= \frac{1}{2}\big[\sin(x + y) + \sin(x - y)\big]\\
			  \tan x \tan y &= \frac{ \tan x + \tan y }{ \cot x + \cot y }\\
			  \tan x \cot y &= \frac{ \tan x + \cot y }{ \cot x + \tan y }
			\end{align*}
		\end{multicols}

\end{document}
