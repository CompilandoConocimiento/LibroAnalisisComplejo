% ****************************************************************************************
% ************************      ANALISIS COMPLEJO             ****************************
% ****************************************************************************************


% =======================================================
% =======         HEADER FOR DOCUMENT        ============
% =======================================================
    % *********   DOCUMENT ITSELF   **************
    \documentclass[12pt, fleqn]{report}                             %Type of docuemtn and size of font and left eq
    \usepackage[margin=1.2in]{geometry}                             %Margins and Geometry pacakge
    \usepackage{ifthen}                                             %Allow simple programming
    \usepackage{hyperref}                                           %Create MetaData for a PDF and LINKS!
    \setlength{\parindent}{0pt}                                     %Eliminate ugly indentation
    \author{Oscar Andrés Rosas}                                     %Who I am

    % *********   LANGUAJE AND UFT-8   *********
    \usepackage[spanish]{babel}                                     %Please use spanish
    \usepackage[utf8]{inputenc}                                     %Please use spanish - UFT
    \usepackage[T1]{fontenc}                                        %Please use spanish
    \usepackage{textcmds}                                           %Allow us to use quoutes
    \usepackage{changepage}                                         %Allow us to use identate paragraphs

    % *********   MATH AND HIS STYLE  *********
    \usepackage{ntheorem, amsmath, amssymb, amsfonts}               %All fucking math, I want all!
    \usepackage{mathrsfs, mathtools, empheq}                        %All fucking math, I want all!
    \usepackage{centernot}                                          %Allow me to negate a symbol
    \decimalpoint                                                   %Use decimal point

    % *********   GRAPHICS AND IMAGES *********
    \usepackage{graphicx}                                           %Allow to create graphics
    \usepackage{wrapfig}                                            %Allow to create images
    \graphicspath{ {Graphics/} }                                    %Where are the images :D

    % *********   LISTS AND TABLES ***********
    \usepackage{listings}                                           %We will be using code here
    \usepackage[inline]{enumitem}                                   %We will need to enumarate
    \usepackage{tasks}                                              %Horizontal lists
    \usepackage{longtable}                                          %Lets make tables awesome
    \usepackage{booktabs}                                           %Lets make tables awesome
    \usepackage{tabularx}                                           %Lets make tables awesome
    \usepackage{multirow}                                           %Lets make tables awesome
    \usepackage{multicol}                                           %Create multicolumns

    % *********   HEADERS AND FOOTERS ********
    \usepackage{fancyhdr}                                           %Lets make awesome headers/footers
    \pagestyle{fancy}                                               %Lets make awesome headers/footers
    \setlength{\headheight}{16pt}                                   %Top line
    \setlength{\parskip}{0.5em}                                     %Top line
    \renewcommand{\footrulewidth}{0.5pt}                            %Bottom line

    \lhead{                                                         %Left Header
        \hyperlink{chapter.\arabic{chapter}}                        %Make a link to the current chapter
        {\normalsize{\textsc{\nouppercase{\leftmark}}}}             %And fot it put the name
    }

    \rhead{                                                         %Right Header
        \hyperlink{section.\arabic{chapter}.\arabic{section}}       %Make a link to the current chapter
            {\footnotesize{\textsc{\nouppercase{\rightmark}}}}      %And fot it put the name
    }

    \rfoot{\textsc{\small{\hyperref[sec:Index]{Ve al Índice}}}}     %This will always be a footer  

    \fancyfoot[L]{                                                  %Algoritm for a changing footer
        \ifthenelse{\isodd{\value{page}}}                           %IF ODD PAGE:
            {\href{https://compilandoconocimiento.com/yo/}          %DO THIS:
                {\footnotesize                                      %Send the page
                    {\textsc{Oscar Andrés Rosas}}}}                 %Send the page
            {\href{https://compilandoconocimiento.com}              %ELSE DO THIS: 
                {\footnotesize                                      %Send the author
                    {\textsc{Compilando Conocimiento}}}}            %Send the author
    }
    
    
    
% ========================================
% ===========   COMMANDS    ==============
% ========================================

    % =====  GENERAL TEXT  ==========
    \newcommand \Quote {\qq}                                        %Use: \Quote to use quotes
    \newcommand \Over {\overline}                                   %Use: \Bar to use just for short
    \newcommand \ForceNewLine {$\Space$\\}                          %Use it in theorems for example
    
    \newenvironment{Indentation}[1][0.75em]                         %Use: \begin{Inde...}[Num]...\end{Inde...}
    {\begin{adjustwidth}{#1}{}}                                     %If you dont put nothing i will use 0.75 em
    {\end{adjustwidth}}                                             %This indentate a paragraph
    \newenvironment{SmallIndentation}[1][0.75em]                    %Use: The same that we upper one, just 
    {\begin{adjustwidth}{#1}{}\begin{footnotesize}}                 %footnotesize size of letter by default
    {\end{footnotesize}\end{adjustwidth}}                           %that's it
        
    % =====  GENERAL MATH  ==========
    \DeclareMathOperator \Space {\quad}                             %Use: \Space for a cool mega space
    \DeclareMathOperator \MiniSpace {\;}                            %Use: \Space for a cool mini space
    \newcommand \Such {\MiniSpace|\MiniSpace}                       %Use: \Such like in sets
    \newcommand \Also {\Space \text{y} \Space}                      %Use: \Also so it's look cool
    \newcommand \Remember[1]{\Space\text{\scriptsize{#1}}}          %Use: \Remember so it's look cool

    \newtheorem{Theorem}{Teorema}[section]                          %Use: \begin{Theorem}[Name]\label{Nombre}...
    \newtheorem{Corollary}{Colorario}[Theorem]                      %Use: \begin{Corollary}[Name]\label{Nombre}...
    \newtheorem{Lemma}[Theorem]{Lemma}                              %Use: \begin{Lemma}[Name]\label{Nombre}...
    \newtheorem{Definition}{Definición}[section]                    %Use: \begin{Definition}[Name]\label{Nombre}...

    \newcommand{\Set}[1]{\left\{ \MiniSpace #1 \MiniSpace \right\}} %Use: \Set {Info}
    \newcommand{\Brackets}[1]{\left[ #1 \right]}                    %Use: \Brackets {Info} 
    \newcommand{\Wrap}[1]{\left( #1 \right)}                        %Use: \Wrap {Info} 
    \newcommand{\pfrac}[2]{\Wrap{\dfrac{#1}{#2}}}                   %Use: Put fractions in parentesis

    \newenvironment{MultiLineEquation}[1]                           %Use: To create MultiLine equations
        {\begin{equation}\begin{alignedat}{#1}}                     %Use: \begin{Multi..}{Num. de Columnas}
        {\end{alignedat}\end{equation}}                             %And.. that's it!
    \newenvironment{MultiLineEquation*}[1]                          %Use: To create MultiLine equations
        {\begin{equation*}\begin{alignedat}{#1}}                    %Use: \begin{Multi..}{Num. de Columnas}
        {\end{alignedat}\end{equation*}}                            %And.. that's it!


    % =====  LOGIC  ==================
    \DeclareMathOperator \doublearrow {\leftrightarrow}             %Use: \doublearrow for a double arrow
    \newcommand \lequal {\MiniSpace \Leftrightarrow \MiniSpace}     %Use: \lequal for a double arrow
    \newcommand \linfire {\MiniSpace \Rightarrow \MiniSpace}        %Use: \lequal for a double arrow
    \newcommand \longto {\longrightarrow}                           %Use: \longto for a long arrow

    % =====  NUMBER THEORY  ==========
    \DeclareMathOperator \Naturals  {\mathbb{N}}                     %Use: \Naturals por Notation
    \DeclareMathOperator \Primes    {\mathbb{P}}                     %Use: \Naturals por Notation
    \DeclareMathOperator \Integers  {\mathbb{Z}}                     %Use: \Integers por Notation
    \DeclareMathOperator \Racionals {\mathbb{Q}}                     %Use: \Racionals por Notation
    \DeclareMathOperator \Reals     {\mathbb{R}}                     %Use: \Reals por Notation
    \DeclareMathOperator \Complexs  {\mathbb{C}}                     %Use: \Complex por Notation

    % === LINEAL ALGEBRA & VECTORS ===
    \DeclareMathOperator \LinealTransformation {\mathcal{T}}        %Use: \LinealTransformation for a cool T

    \newcommand{\pVector}[1]{                                       %Use: \pVector {Matrix Notation} use parentesis
        \ensuremath{\begin{pmatrix}#1\end{pmatrix}}                 %Example: \pVector{a\\b\\c} or \pVector{a&b&c} 
    }
    \newcommand{\lVector}[1]{                                       %Use: \lVector {Matrix Notation} use a abs 
        \ensuremath{\begin{vmatrix}#1\end{vmatrix}}                 %Example: \lVector{a\\b\\c} or \lVector{a&b&c} 
    }
    \newcommand{\bVector}[1]{                                       %Use: \bVector {Matrix Notation} use a brackets 
        \ensuremath{\begin{bmatrix}#1\end{bmatrix}}                 %Example: \bVector{a\\b\\c} or \bVector{a&b&c} 
    }
    \newcommand{\Vector}[1]{                                        %Use: \Vector {Matrix Notation} no parentesis
        \ensuremath{\begin{matrix}#1\end{matrix}}                   %Example: \Vector{a\\b\\c} or \Vector{a&b&c}
    }

    % MATRIX
    \makeatletter                                                   %Example: \begin{matrix}[cc|c]
    \renewcommand*\env@matrix[1][*\c@MaxMatrixCols c] {             %WTF! IS THIS
        \hskip -\arraycolsep                                        %WTF! IS THIS
        \let\@ifnextchar\new@ifnextchar                             %WTF! IS THIS
        \array{#1}                                                  %WTF! IS THIS
    }                                                               %WTF! IS THIS
    \makeatother                                                    %WTF! IS THIS

    % TRIGONOMETRIC FUNCTIONS
    \newcommand{\Cos}[1]{\cos\Wrap{#1}}
    \newcommand{\Sin}[1]{\sin\Wrap{#1}}

    % === COMPLEX ANALYSIS ===
    \newcommand \Cis[1]  {\Cos{#1} + i \Sin{#1}}                    %Use: \Cis for cos(x) + i sin(x)
    \newcommand \pCis[1] {\Wrap{\Cis{#1}}}                          %Use: \pCis for the same ut parantesis



% =====================================================
% ============     	  COVER PAGE	   ================
% =====================================================
\begin{document}
\begin{titlepage}

	\center
	% ============ UNIVERSITY NAME AND DATA =========
	\textbf{\textsc{\Large Proyecto Compilando Conocimiento}}\\[1.0cm] 
	\textsc{\Large Matemáticas Avanzadas}\\[1.0cm] 

	% ============ NAME OF THE DOCUMENT  ============
	\rule{\linewidth}{0.5mm} \\[1.0cm]
		{ \huge \bfseries Análisis Complejo}\\[1.0cm] 
	\rule{\linewidth}{0.5mm} \\[2.0cm]
	
	% ====== SEMI TITLE ==========
	{\LARGE Una Pequeña (Gran) Introducción}\\[7cm] 
	
	% ============  MY INFORMATION  =================
	\begin{center} \large
	\textbf{\textsc{Autores:}}\\
	Rosas Hernandez Oscar Andrés \\
	Lopez Manriquez Angel
	\end{center}

	\vfill

\end{titlepage}

% =====================================================
% ========                INDICE              =========
% =====================================================
\tableofcontents{}
\label{sec:Index}

\clearpage




% //////////////////////////////////////////////////////////////////////////////////////////////////////////
% ///////////////////////////////////         NUMEROS COMPLEJOS       //////////////////////////////////////
% //////////////////////////////////////////////////////////////////////////////////////////////////////////
\part{Números Complejos}
\clearpage


    % ===============================================================================
    % ===================           DEFINICIONES               ======================
    % ===============================================================================
    \chapter{Definiciones}

        % ==============================================
        % ========      NUMEROS COMPLEJOS      =========
        % ==============================================
        \clearpage
        \section{Definición de Números Complejos}

            \begin{Definition}[Números Complejos]
            \label{NumerosComplejos}
                Definamos al Conjunto de los números complejos $\Complexs$ como:
                \begin{equation}
                    \Complexs = 
                        \Set{ a + bi \Such a,b \in \Reals \Also i = \sqrt{-1} }
                \end{equation}

            \end{Definition}


            Podemos usar la notación $a+bi$, $a+ib$ y $(a, b)$ de manera intercambiable (pero personalmente 
            la primera se me hace la más cool pero la ultima mas concreta).


        % ==============================================
        % ========      TERMINOS COMUNES       =========
        % ==============================================
        \section{Términos Comúnes} 

            \begin{itemize}
                \item \textbf{Unidad Imaginaria:}
                    Usamos el símbolo $i$ para simplificar $i = \sqrt{-1}$, de ahí la propiedad
                    famosa $i^2 = -1$.

                \item \textbf{Parte Real:}
                    Considere el complejo $z = a+bi \in \Complexs$, entonces decimos que $Re(z) = a$

                \item \textbf{Parte Imaginaria:}
                    Considere el complejo $z = a+bi \in \Complexs$, entonces decimos que $Im(z) = b$
            \end{itemize}



    % ===============================================================================
    % ===================          ARITMETICA COMPLEJA         ======================
    % ===============================================================================
    \chapter{Aritmética Compleja}

        % ==============================================
        % ========      NUMEROS COMPLEJOS      =========
        % ==============================================
        \clearpage
        \section{Operaciones Básicas}
            Si $z_1 = a_1 + b_1i \in \Complexs$ y $z_2 = a_2 + b_2i \in \Complexs$ entonces:

            \begin{itemize}

                \item
                    \begin{Definition}[Suma de Complejos]
                    \label{SumaComplejos}
                        \begin{equation}
                            z_1 + z_2 = (a_1+a_2) + (b_1+b_2)i
                        \end{equation}
                    \end{Definition}

                \item
                    \begin{Definition}[Resta de Complejos]
                    \label{RestaComplejos}
                        \begin{equation}
                            z_1 - z_2 = (a_1-a_2) + (b_1-b_2)i
                        \end{equation}
                    \end{Definition}

                \item 
                    \begin{Definition}[Multiplicación de Complejos]
                    \label{MultiplicacionComplejos}

                        \begin{MultiLineEquation}{1}
                            z_1 z_2 &= (a_1+b_1i)(a_2+b_2i) 
                                     = (a_1a_2 + b_1b_2i^2) + (a_1b_2 + b_1a_2)i  \\
                                    &= (a_1a_2 - b_1b_2)   + (a_1b_2 + b_1a_2)i
                        \end{MultiLineEquation}

                    \end{Definition}

                \item 
                    \begin{Definition}[División de Complejos]
                    \label{DivisionComplejos}
                    \ForceNewLine
                        \begin{MultiLineEquation}{1}
                            \dfrac{z_1}{z_2}    &= \dfrac{a_1+b_1i}{a_2+b_2i} 
                                                &= \dfrac{(a_1a_2+b_1b_2)-(a_1b_2-a_2b_1)i}{(a_2)^2+(b_2)^2} \Space z_2 \neq 0
                        \end{MultiLineEquation}

                    \end{Definition}

            \end{itemize}


        % =====================================================
        % ==========   CAMPO DE LOS COMPLEJOS      ============
        % =====================================================
        \clearpage
        \section{Campo de los Complejos}

            Recuerda que el hecho de que los Complejos sean un campo nos dice que cumple con que:

            \begin{itemize}
                    
                \item 
                    \begin{Definition}[Ley Aditiva Asociativa]
                        \begin{equation}
                            \forall z_1, z_2, z_3 \in \Complexs, \MiniSpace
                                (z_1 + z_2) + z_3 = z_1 + (z_2 + z_3)
                        \end{equation}
                    \end{Definition}

                \item
                    \begin{Definition}[Ley Aditiva Conmutativa]
                        \begin{equation}
                            \forall z_1, z_2 \in \Complexs, \MiniSpace z_1 + z_2 =  z_2 + z_1
                        \end{equation}
                    \end{Definition}

                \item
                    \begin{Definition}[Elemento Indentidad Aditivo]
                        \begin{equation}
                            \exists 0 \in \Complexs, \MiniSpace
                                \forall z_1 \in \Complexs, \MiniSpace 0 + z_1 = z_1  + 0 = z_1
                        \end{equation}
                    \end{Definition}


                \item
                    \begin{Definition}[Existen Inversos Aditivos]
                        \begin{equation}
                            \forall z_1 \in \Complexs, \MiniSpace
                                \exists z_2 \in \Complexs, \MiniSpace
                                    z_1  + z_2 = z_2 + z_1 = 0
                        \end{equation}
                    \end{Definition}

                \item
                    \begin{Definition}[Ley Distributiva]
                        \begin{MultiLineEquation}{1}
                            \forall z_1, z_2, z_3 \in \Complexs, \MiniSpace
                                z_1 \cdot (z_2  + z_3) = (z_1  \cdot z_2) + (z_1  \cdot z_3)        \\
                            \forall z_1, z_2, z_3 \in \Complexs, \MiniSpace
                                (z_2 + z_3) \cdot z_1  = (z_2 \cdot z_1) + (z_3  \cdot z_1)
                        \end{MultiLineEquation}
                    \end{Definition}


                \item 
                    \begin{Definition}[Ley Multiplicativa Asociativa]
                        \begin{equation}
                            \forall z_1, z_2, z_3 \in \Complexs, \MiniSpace
                                z_1  \cdot z_2 = z_2 \cdot z_1
                        \end{equation}
                    \end{Definition}

                \item 
                    \begin{Definition}[Ley Multiplicativa Distributiva]
                        \begin{equation}
                            \forall z_1, z_2, z_3 \in \Complexs, \MiniSpace
                                (z_1  \cdot z_2)  \cdot z_3 = z_1  \cdot (z_2  \cdot z_3)
                        \end{equation}
                    \end{Definition}

                \item
                    \begin{Definition}[Elemento Indentidad Multiplicativo]
                        \begin{equation}
                            \exists 1 \in \Complexs, \MiniSpace
                                \forall z_1 \in \Complexs, \MiniSpace 1 \cdot z_1 = z_1  \cdot 1 = z_1
                        \end{equation}
                    \end{Definition}

                \item
                    \begin{Definition}[Existen Inversos Multiplicativos]
                        \begin{equation}
                            \forall z_1 \in \Complexs - \{0\}, \MiniSpace
                                \exists z_2 \in \Complexs, \MiniSpace
                                    z_1  + z_2= z_2 + z_1 = 1
                        \end{equation}
                    \end{Definition}

                \end{itemize}


        % ==============================================
        % ========      CERO Y LA IDENTIDAD    =========
        % ==============================================
        \clearpage
        \section{Cero y la Identidad}

            \begin{itemize}
                \item Denotamos a $0 = 0 + 0i$ como el elemento cero o identidad aditiva, ya que se cumple 
                        $\forall z \in \Complexs, \MiniSpace z + 0 = 0 + z = z$

                \item Denotamos a $1 = 1 + 0i$ como el elemento identidad multiplicatica, ya que se cumple 
                        $\forall z \in \Complexs, \MiniSpace z \cdot 1 = 1 \cdot z = z$
            \end{itemize}

        % ==============================================
        % ========          CONJUGADO          =========
        % ==============================================
        \section{Conjugados}
            Tenemos que el Conjugado de $z = a+bi \in \Complexs$ es simplemente $\overline{z} = a-bi$

            Además tenemos que:
            \begin{itemize}
                \item $\overline{z_1 + z_2} = \overline{z_1} + \overline{z_2}$
                
                \item $\overline{z_1 \cdot z_2} = \overline{z_1} \cdot \overline{z_2}$

                \item $\overline{\Pi_{j = 1}^n {z_j}} = \prod_{j = 1} ^ n \overline{z_j} $
                
                \item $\overline{\overline{z}} = z$
                
                \item $z \cdot \overline{z} = (a+ib)(a-ib) = a^2 + b^2 = |z| ^ 2$

                \item $Re(z) = \dfrac{z+\overline{z}}{2}$

                \item $Im(z) = \dfrac{z-\overline{z}}{2i}$
            \end{itemize}



        % ==============================================
        % ========    MODULO O VALOR ABSOLUTO    =======
        % ==============================================
        \clearpage
        \section{Módulo o Valor Absoluto}
            Tenemos que el Módulo de $z = a+bi \in \Complexs$ es simplemente $|z| = \sqrt{a^2 + b^2}$.

            \begin{itemize}
                \item $|z| = \sqrt{z \cdot \overline{z}}$
                \item $|z| = \sqrt{a^2 + b^2} = \sqrt{a^2 + (-b)^2} = |\Over{z}|$
            \end{itemize}

            \begin{itemize}
                \item
                    \begin{Lemma}
                        $|Re(z)| \leq |z|$ y $|Im(z)| \leq |z|$ 

                        % ======== DEMOSTRACION ========
                        \begin{SmallIndentation}[1em]
                            \textbf{Demostración}:
                            
                            Ya habiamos visto que $|z|^2 = x^2 + y^2 = Re(z)^2 + Im(z)^2$
                            
                            Entonces podemos ver que $|z|^2 - Im(z)^2 = Re(z)$ (recuerda que $Im(z)^2 > 0$) 
                            por lo tanto tenemos que $|Re(z)|^2 \leq |z|^2$ ya que $|Re(z)| = Re(z)$
                            
                            Entonces podemos ver que $|z|^2 - Re(z)^2 = Im(z)$ (recuerda que $Re(z)^2 > 0$) 
                            por lo tanto tenemos que $|Im(z)|^2 \leq |z|^2$ ya que $|Im(z)| = Im(z)$

                        \end{SmallIndentation}

                    \end{Lemma}

                \item
                    \begin{Lemma}
                        $|z_1+z_2|^2 = |z_1|^2 + |z_2|^2 + 2Re(z_1\Over{z_2})$ 

                        % ======== DEMOSTRACION ========
                        \begin{SmallIndentation}[1em]
                            \textbf{Demostración}:

                            Ya sabemos que $|z+\Over{z}|^2 = z \cdot \Over{z}$ y recuerda que
                            $2Re(z) = z+\overline{z}$, $|z|^2 = z \cdot \overline{z}$ entonces tenemos que:
                            \begin{MultiLineEquation*}{2}
                                |z_1+z_2|^2 &= (z_1+z_2) \cdot \Over{(z_1+z_2)}                                 \\
                                        &= (z_1+z_2) \cdot (\Over{z_1}+\Over{z_2})                              \\
                                        &= z_1 \bar{z_1} + (z_1\Over{z_2}+\Over{z_1}z_2) + z_2 \Over{z_2}       \\
                                        &= z_1 \bar{z_1} + (z_1\Over{z_2}+\Over{z_1\Over{z_2}})+z_2 \Over{z_2}  \\
                                        &= |z_1|^2 + 2Re(z_1\Over{z_2}) + |z_2|^2 
                            \end{MultiLineEquation*}
                            
                        \end{SmallIndentation}

                    \end{Lemma}


                \item
                    \begin{Lemma}
                        $(|z_1|+|z_2|)^2 = |z_1|^2 + |z_2|^2 + 2|z_1\Over{z_2}|$ 

                        % ======== DEMOSTRACION ========
                        \begin{SmallIndentation}[1em]
                            \textbf{Demostración}:

                            Esta la vamos a empezar al réves, solo recuerda que $|z|=|\Over{z}|$:
                            \begin{MultiLineEquation*}{2}
                                |z_1|^2 + |z_2|^2 + 2|z_1\Over{z_2}|                                  
                                        &= |z_1|^2 + |z_2|^2 + 2|z_1||\Over{z_2}|                     \\
                                        &= |z_1|^2 + |z_2|^2 + 2|z_1||z_2|                            \\
                                        &= (|z_1| + |z_2|)^2
                            \end{MultiLineEquation*}
                            
                        \end{SmallIndentation}

                    \end{Lemma}

                \clearpage
                \item
                    \begin{Lemma}[Desigualdad del Triángulo]
                    \label{DesiguldadTriangulo}
                        $|z_1|-|z_2| \leq |z_1+z_2| \leq |z_1|+|z_2|$

                        % ======== DEMOSTRACION ========
                        \begin{SmallIndentation}[1em]
                            \textbf{Demostración}:


                            Ok, esto aún estará intenso, así que sígueme, vamos a hacerlo más
                            interesante, ya tenemos las piezas necesarias.
                            Así que vamos a hacerlo al réves:

                            $|z_1+z_2| \leq |z_1|+|z_2|$ si y solo si 
                            $|z_1+z_2|^2 = (|z_1|+|z_2|)^2$ y además $|z_1+z_2|,|z_1|,|z_2| \geq 0$
                            lo cual si que se cumple, pues los módulos nunca son negativos.

                            Y lo que dije anteriormente se cumple si y solo si $|z_1+z_2|^2=(|z_1|+|z_2|)^2+k$
                            donde $k \geq 0$.

                            Ya sabemos que $|z_1+z_2|^2 = |z_1|^2 + |z_2|^2 + 2Re(z_1\Over{z_2})$
                            y $(|z_1|+|z_2|)^2 = |z_1|^2 + |z_2|^2 + 2|z_1\Over{z_2}|$, ahora vamos a acomodar
                            un poco, podemos poner lo último como
                            $(|z_1|+|z_2|)^2  - 2|z_1\Over{z_2}| = |z_1|^2 + |z_2|^2$

                            Ahora veamos que:
                            \begin{MultiLineEquation*}{2}
                                |z_1+z_2|^2 &= \Brackets{|z_1|^2 + |z_2|^2} + 2Re(z_1\Over{z_2})  
                                             = \Brackets{(|z_1|+|z_2|)^2-2|z_1\Over{z_2}|} + 2Re(z_1\Over{z_2}) \\
                                            &= (|z_1|+|z_2|)^2 + k
                            \end{MultiLineEquation*}

                            Donde $k = 2Re(z_1\Over{z_2})- 2|z_1\Over{z_2}|$, ahora además podemos decir
                            que si $k \geq 0$ entonces así lo será $\frac{k}{2}$, por lo tanto:
                            $\frac{k}{2} = Re(z_1\Over{z_2}) - |z_1\Over{z_2}|$, pero si les cambias en nombre
                            ves que todo se simplifica $w = z_1\Over{z_2}$ y tenemos que $Re(w) - |w|$.
                            Espera, recuerda que ya habíamos demostrado que $|Re(z)| \leq |z|$, así que por lo
                            tanto $k \geq 0$ y la propiedad siempre se cumple.

                            Sabemos que $z_1 = z_1 + z_2 + (-z_2)$ además ahora sabemos que:
                            $|z_1| = |z_1 + z_2 +(-z_2)| \leq |z_1 + z_2| + |-z_2|$ y como $|z|=|-z|$
                            Que es lo mismo que $|z_1| - |z_2| \leq |z_1 + z_2|$.

                            Y listo, todas las propiedades están listas.


                            Además creo que es bastante obvio que por inducción tenemos que:
                            $|z_1 + z_2 +z_3 + \cdots + z_n| \leq |z_1|+|z_2|+|z_3|+ \cdots +|z_n|$

                        \end{SmallIndentation}

                    \end{Lemma}




            \end{itemize}



        % ==============================================
        % ========      NUMEROS COMPLEJOS      =========
        % ==============================================
        \clearpage
        \section{Inverso Multiplicativo}
            
            Si $z = a + bi \in \Complexs - \Set{0}$ entonces podemos denotar al inverso de $z$ como
            $z^{-1}$

            Creo que es más que obvio que $z^{-1} = \dfrac{1}{a+bi}$.

            \begin{itemize}

                \item
                    Podemos escribir a $z^{-1}$ como $\dfrac{a-ib}{a^2+b^2}$
                    % ======== DEMOSTRACION ========
                    \begin{SmallIndentation}[1em]
                        \textbf{Demostración}:
                        
                        Veamos como llegar a eso paso a paso:
                        \begin{MultiLineEquation*}{1}
                            \dfrac{1}{z} &= \dfrac{1}{a+bi}                         
                                          = \dfrac{1}{a+bi}\pfrac{a-bi}{a-bi}
                                          = \dfrac{a-bi}{(a+bi)(a-bi)}              \\
                                         &= \dfrac{a-bi}{a^2+b^2}                 
                        \end{MultiLineEquation*}

                    \end{SmallIndentation}



                \item
                    Gracias al inciso anterior podemos separar la parte real y la imaginaria como:
                    \begin{equation}
                         \dfrac{1}{z} = \pfrac{a}{a^2+b^2} - \pfrac{b}{a^2+b^2}i 
                    \end{equation}


                \item
                    Gracias al inciso anterior podemos separar la parte real y la imaginaria como:
                    \begin{equation}
                        \dfrac{1}{z} = \dfrac{\overline{z}}{|z|^2}
                    \end{equation}

                    % ======== DEMOSTRACION ========
                    \begin{SmallIndentation}[1em]
                        \textbf{Recuerda}: $\overline{z} = a - bi$ y $|z| = a^2 + b^2$
                    \end{SmallIndentation}

                

            \end{itemize}



        % ==============================================
        % ========      RAICES DE UN NUMERO N     ======
        % ==============================================
        \clearpage
        \section{n-Raíces de un Numero $z$}

            En general decir que un número $w$ es un raíz enesíma de un número complejo $z$ (siempre
            que $z \neq 0$) si cumpla que:
            \begin{equation}
                w^n = z
            \end{equation}
            Donde obviamente $n \in \Integers^+$.


            \begin{Theorem}{Existen exactamente $n$ raíces para $w^n = z$}

                Y puedes encontrar las $n$ raíces variando $k$ de 0 a $n$:
                \begin{equation}
                    w_k = n\sqrt{r} \Cis{\dfrac{\theta + (2\pi) k}{n}}
                \end{equation}

                % ======== DEMOSTRACION ========s
                    \begin{SmallIndentation}[1em]
                        \textbf{Demostración}:
                        
                        Tengamos dos números:
                        \begin{itemize}
                            \item $z = r\pCis{\theta}$
                            \item $w = p\pCis{\phi}$
                        \end{itemize}

                        Entonces de la ecuación:
                        \begin{MultiLineEquation*}{3}
                            w^n &= z  \\ 
                            \Brackets{p\pCis{\phi}}^n &= r\pCis{\theta}
                                &&  \Remember{Simplemente sustituimos}                              \\
                        \end{MultiLineEquation*}

                        Tenemos que:
                        \begin{itemize}
                            \item $p^n = r$\\
                                
                                Por lo tanto podemos definir a $p = \sqrt{n}$ donde $\sqrt{n}$
                                es la única raíz positiva de un número $r \in \Reals$.

                            \item $\Cis{\theta}^n = \Cis{\phi}$\\

                                Gracias por $\Cis{n \theta} = \Cis{\phi}$
                                Por lo tanto podemos decir que:
                                \begin{itemize}
                                    \item $\Cos{n \theta} = \Cos{\phi}$
                                    \item $\Sin{n \theta} = \Sin{\phi}$
                                \end{itemize}

                                Por $\phi = \dfrac{\theta + 2(n + m) \pi}{n} = \dfrac{\theta + 2mn\pi}{n} + 2\pi$
                                y por lo tanto tener que:
                                \begin{itemize}
                                    \item $\Sin{\phi} = \Sin{\dfrac{\theta + (2\pi) k}{n}}$
                                    \item $\Cos{\phi} = \Cos{\dfrac{\theta + (2\pi) k}{n}}$
                                \end{itemize}

                        \end{itemize}


                        Por lo tanto podemos generalizar los resultados como:
                        \begin{equation}
                            w_k = n\sqrt{r} \Cis{\dfrac{\theta + (2\pi) k}{n}}
                        \end{equation}


                    \end{SmallIndentation}
            \end{Theorem}



    % ===============================================================================
    % ===================              FORMA POLAR             ======================
    % ===============================================================================
    \chapter{Forma Polar y Argumentos}

        % ==============================================
        % ========         FORMA POLAR         =========
        % ==============================================
        \clearpage
        \section{Forma Polar}
            
            Podemos expresar un punto en el plano complejo mediante la tupla $(r, \theta)$ , donde
            $r \geq 0$ y $\theta$ esta medido en radianes.

            Entonces podemos pasar rápido y fácil de un sistema de coordenadas a otro como:

            % ==============================
            % ===  POLAR -> RECTANGULAR  ===
            % ==============================
            \subsection{De forma Polar a forma Rectangular}

                Supongamos que tenemos un punto que podemos describir como $(r, \theta)$,
                donde $r \geq 0$ y $\theta$ medido como radianes.

                Entonces tenemos que:

                \begin{itemize}
                     \item $a = r \Cos{\theta}$
                     \item $b = r \Sin{\theta}$
                 \end{itemize}

                 Otra forma de escribirlo es $r(\Cos{\theta} + i\Sin{\theta})$

            % ==============================
            % ===  RECTANGULAR -> POLAR  ===
            % ==============================
            \subsection{De forma Rectangular a forma Polar}

                Supongamos que tenemos un punto que podemos describir como $(a+bi)$,
                entonces podemos decir que:

                \begin{itemize}
                    \item $r = \sqrt{a^2+b^2}$
                    \item $\theta = \begin{cases}
                                        \tan(\frac{b}{a})^{-1}      \Space &\text{ si } a > 0                   \\
                                        \tan(\frac{b}{a})^{-1} +\pi \Space &\text{ si } a < 0 \text{ y } b > 0  \\
                                        \tan(\frac{b}{a})^{-1} -\pi \Space &\text{ si } a < 0 \text{ y } b < 0  \\
                                    \end{cases}$
                \end{itemize}

        % ==============================================
        % ========         ARGUMENTOS          =========
        % ==============================================
        \clearpage
        \section{Argumento de $z$}
            
            Definimos al argumento de un número $z = a+bi \in \Complexs$ como $\theta = arg(z)$,
            es decir, al final del día $arg(z)$ es un ángulo.

            Este ángulo tiene que cumplir las dos siguientes ecuaciones:

            \begin{itemize}
                \item $\Cos{\theta} = \dfrac{x}{\sqrt{a^2+b^2}}$
                \item $\Sin{\theta}   = \dfrac{y}{\sqrt{a^2+b^2}}$
            \end{itemize}

            Pero como $\sin$ y $\cos$ con funciones periodicas con $2\pi$, es decir $arg(z)$ no es único.

            Además para encontrarlo usamos $\tan(\frac{b}{a})^{-1}$ pero resulta que esta función solo
            regresa ángulos entre $-\frac{\pi}{2}$ y $\frac{\pi}{2}$ por lo tanto habrá problemas con
            números en el segundo y tercer cuadrante.

            % =================================
            % ====   ARGUMENTO PRINCIPAL  =====
            % =================================
            \subsection*{Argumento Principal}

                Ya que $arg(z)$ es más bien un conjunto de ángulos, podemos considerar al ángulo o 
                argumento principal de $z$ como $Arg(z)$ y que será el ángulo que cumpla con que:

                \begin{itemize}
                    \item $\Cos{Arg(z)} = \dfrac{x}{\sqrt{a^2+b^2}}$
                    \item $\Sin{Arg(z)}   = \dfrac{y}{\sqrt{a^2+b^2}}$
                    \item $-\frac{\pi}{2} < Arg(z) \leq \frac{\pi}{2}$
                \end{itemize}

                Podemos probar que $Arg(z)$ para alguna $z$ cualquiera será única.

                Por lo tanto ahora podemos definir a $arg(z)$ como:
                \begin{equation}
                    arg(z) = \Set{ Arg(z) + 2n\pi \Such n \in \Integers }
                \end{equation}


        % ==============================================
        % =======     LEYES DE ARITMETICA      =========
        % ==============================================
        \clearpage
        \section{Leyes de Aritmetica}
            
            Supón dos números complejos de manera polar como $z_1 = (r_1, \theta_1)$ y $z_1 = (r_2, \theta_2)$ 
            es decir $z_1 = r_1(\Cos{\theta_1} + i\Sin{\theta_1}$ y
            $z_2 = r_2(\Cos{\theta_2} + i\Sin{\theta_2}$ entonces tenemos que:

            \begin{itemize}
                \item
                    \textbf{Producto de Números Complejos:} \\
                    $z_1z_2 = [(r_1r_2), (\theta_1 + \theta_2)]$

                    % ======== DEMOSTRACION ========
                    \begin{SmallIndentation}[1em]
                        \textbf{Demostración}:

                        Esto es muy sencillo, primero ya que tenemos los dos números en forma rectangular
                        podemos multiplicar como ya sabemos:
                        \begin{MultiLineEquation*}{2}
                            z_1 z_2 &= (a_1a_2 - b_1b_2) + (a_1b_2 + b_1a_2)i \\
                            z_1 z_2 &= r_1r_2[(\Cos{\theta_1}\Cos{\theta_2} - \Sin{\theta_1}\Sin{\theta_2}) 
                                        + (\Cos{\theta_1}\Sin{\theta_2} + \Sin{\theta_1}\Cos{\theta_2})i]
                        \end{MultiLineEquation*}

                        Usando las leyes de senos y cosenos:
                        \begin{itemize}
                            \item $\Cos{a\pm b} = \Cos{a}\Cos{b} \mp \Sin{a}\Sin{b}$
                            \item $\Sin{a\pm b} = \Sin{a}\Cos{b} \pm \Cos{a}\Sin{b}$
                        \end{itemize}

                        Podemos reducirlo a:
                        $z_1 z_2 = r_1r_2 [\Cos{\theta_1 + \theta_2} + i\Sin{\theta_1 + \theta_2}]$
                        y creo que de ahí podemos reducirlo casi mentalmente ya que 
                        $(r, \theta) = r(\Cos{\theta} + i \Sin{\theta})$

                    \end{SmallIndentation}

                \item
                    \textbf{División de Números Complejos:} \\
                    $\frac{z_1}{z_2} = [(\frac{r_1}{r_2}), (\theta_1 - \theta_2)]$

                    % ======== DEMOSTRACION ========
                    \begin{SmallIndentation}[1em]
                        \textbf{Demostración}:

                        Esto es muy sencillo, primero ya que tenemos los dos números en forma rectangular
                        podemos dividir como ya sabemos, pero vamos a hacer un poco de trampa ingeniosa,
                        usamos la idea de que $\dfrac{1}{z} = \dfrac{\overline{z}}{|z|^2}$ y hacer:

                        \begin{MultiLineEquation*}{2}
                            \frac{z_1}{z_2} &= z_1 \dfrac{\overline{z_2}}{|z_2|^2} 
                                             = z_1 \dfrac{\overline{z_2}}{(r_2)^2}                          \\
                                            &= \frac{1}{(r_2)^2} z_1 \overline{z_2} 
                                             = \frac{1}{(r_2)^2} (a_1 + ib_1) (a_2 - ib_2)                  \\
                                            &= \frac{1}{(r_2)^2} (a_1a_2 - b_1b_2) + (a_1b_2 - b_1a_2)i     \\
                                            &= \frac{r_1}{r_2}
                                                [(\Cos{\theta_1}\Cos{\theta_2} +  \Sin{\theta_1}\Sin{\theta_2}) 
                                                + (\Cos{\theta_1}\Sin{\theta_2} - \Sin{\theta_1}\Cos{\theta_2})i]
                        \end{MultiLineEquation*}

                        Usando las leyes de senos y cosenos:
                        \begin{itemize}
                            \item $\Cos{a\pm b} = \Cos{a}\Cos{b} \mp \Sin{a}\Sin{b}$
                            \item $\Sin{a\pm b} = \Sin{a}\Cos{b} \pm \Cos{a}\Sin{b}$
                        \end{itemize}

                        Podemos reducirlo a:
                        $z_1 z_2 = \frac{r_1}{r_2} [\Cos{\theta_1 - \theta_2} + i\Sin{\theta_1 - \theta_2}]$
                        y creo que de ahí podemos reducirlo casi mentalmente ya que 
                        $(r, \theta) = r(\Cos{\theta} + i \Sin{\theta})$

                    \end{SmallIndentation}


                \item
                    \textbf{Simplificar Potencias de $z$:} \\
                    
                    $z^n = [(r^n), (n \cdot \theta)]$

                    % ======== DEMOSTRACION ========
                    \begin{SmallIndentation}[1em]
                        \textbf{Ideas}:

                         No considero a esto una demostración, pero si te das cuenta usando
                         $z^2 = [(r^2), (\theta + \theta)]$.
 
                         Y cosas como $z^{-2} = [(r^{-2}), (-\theta -\theta)]$
                         
                    \end{SmallIndentation}


            \clearpage
            

            \end{itemize}


        % ==============================================
        % ========         LEYES               =========
        % ==============================================
        \clearpage
        \section{Ley de Moivre's}

            $z^n = r^n \Wrap{\Cis{n \cdot \theta} \Space \text{donde } n \in \Integers$
            
            % ======== DEMOSTRACION ========
            \begin{SmallIndentation}[1em]
                \textbf{Demostración}:

                Se puede dar una demostracion muy sencilla, no se porque los libros usan induccion matematica para
                demostrar el teorema de Moivre... en fin, expresando a $z$ en su forma polar y usando la formula de Euler, tenemos: 

                \begin{MultiLineEquation}{2}
				  z ^ n &= \Wrap{r \Cis{\theta}}^n              \\
                        &= r^n \Wrap{\Cis{\theta}^n             \\
				  		&= r^n e^{n\theta i}                    \\
                        &= r^n \Wrap{\Cis{n \cdot \theta}	    \\ 
				\end{align}
            \end{MultiLineEquation}



    % ===============================================================================
    % ===================         FORMA EXPONENCIAL             =====================
    % ===============================================================================
    \chapter{Forma Exponencial}


        % ==============================================
        % =====      FORMULAR DE EULER             =====
        % ==============================================
        \clearpage
        \section{Fórmula de Euler}  

            \[ e^{i\theta} = \Cos{\theta} + i\Sin{\theta} \]

            % ======== DEMOSTRACION ========
            \begin{SmallIndentation}[1em]
      
                Esta formula sale de a partir de la serie de McLaurin para la funcion exponencial

                \[e^k = 1 + \dfrac{k}{1!} + \dfrac{k^2}{2!} + \dfrac{k^3}{3!} + \cdots\]

                Pasa algo muy interesante al hacer $k = i\theta$, pues vemos que se hayan las series de
                McLaurin del seno y coseno:

                \begin{MultiLineEquation*}{2}
                    e^{i\theta} &= 
                                \Wrap{1 - \dfrac{\theta^2}{2!} + \dfrac{\theta^4}{4!} + \cdots}
                                +
                                i\Wrap{\theta - \dfrac{\theta^3}{3!} + \dfrac{\theta^5}{5!} + \cdots} \\
                                &= \Cos{\theta} + i\Sin{\theta}
                \end{MultiLineEquation*}

            \end{SmallIndentation}

        


        % ==============================================
        % =====     IDENTIDAD DE LAGRANGE          =====
        % ==============================================
        \clearpage
        \section{Identidad de Lagrange}

            \begin{equation}
                1+\Cos{1\theta}+\Cos{2\theta}+\cdots+\Cos{n\theta}  = 
                    \dfrac{1}{2} \Wrap{\dfrac{\Sin{(n+\frac{1}{2})\theta}}{\Sin{\dfrac{\theta}{2}}}+1}
            \end{equation}
                
            % ======== DEMOSTRACION ========
            \begin{SmallIndentation}[1em]
              
                \begin{MultiLineEquation*}{3}
                    \sum_{k=0}^{n} \Cos{k\theta} 
                        &= \dfrac{1}{2} \sum_{k=0}^{n}\Wrap{e^{ik\theta}+e^{-ik\theta}}
                        &&  \Remember{Recuerda que: $\Cos{x} = \dfrac{e^{ik\theta}+e^{-ik\theta}}{2}$}   \\
                        &= \dfrac{1}{2} \Wrap{
                                            \dfrac{e^{(n+1)i\theta} - 1}{e^{i\theta} - 1}   +
                                            \dfrac{e^{(n+1)-i\theta} - 1}{e^{-i\theta} - 1} }
                        &&  \Remember{Recuerda que: $\sum_{k=0}^n r^k = \dfrac{r^{n+1}-1}{r-1}  \iff |r| < 1$}        \\
                        &= \dfrac{1}{2} \Wrap{
                                            \dfrac{e^{(n+1)i\theta} - 1}{e^{i\theta} - 1}
                                                \pfrac{e^{-i\frac{\theta}{2}}}{e^{-i\frac{\theta}{2}}}
                                            +
                                            \dfrac{e^{(n+1)-i\theta} - 1}{e^{-i\theta} - 1}
                                                \pfrac{-e^{i\frac{\theta}{2}}}{-e^{i\frac{\theta}{2}}}
                                        }
                        &&  \Remember{Multiplica por Uno ;)}                                            \\
                        &= \dfrac{1}{2} \Wrap{
                                            \dfrac{e^{(n+\frac{1}{2})i\theta} - e^{i\frac{\theta}{2}}}
                                                  {e^{i\frac{\theta}{2}} - e^{-i\frac{\theta}{2}}}
                                            +
                                            \dfrac{e^{i\frac{\theta}{2}} - e^{(n+\frac{1}{2})-i\theta}}
                                                  {e^{i\frac{\theta}{2}} - e^{-i\frac{\theta}{2}}}
                                        }
                        &&  \Remember{Expande}                                                          \\
                        &= \dfrac{1}{2} \Wrap{
                                            \dfrac{e^{(n+\frac{1}{2})i\theta} - e^{(n+\frac{1}{2})-i\theta}}
                                                  {e^{i\frac{\theta}{2}} - e^{-i\frac{\theta}{2}}}
                                            +
                                            \dfrac{e^{i\frac{\theta}{2}} - e^{-i\frac{\theta}{2}}}
                                                  {e^{i\frac{\theta}{2}} - e^{-i\frac{\theta}{2}}}
                                        }
                        &&  \Remember{Organizando, gracias denominador común}                           \\
                        &= \dfrac{1}{2} \Wrap{
                                            \dfrac{e^{(n+\frac{1}{2})i\theta} - e^{(n+\frac{1}{2})-i\theta}}
                                                  {e^{i\frac{\theta}{2}} - e^{-i\frac{\theta}{2}}}
                                            +
                                            1
                                        }
                        &&  \Remember{Simplificar}                                                      \\
                        &= \dfrac{1}{2} \Wrap{
                                            \dfrac{
                                                \dfrac{
                                                        e^{(n+\frac{1}{2})i\theta}
                                                           - 
                                                        e^{(n+\frac{1}{2})-i\theta}
                                                      }
                                                      {2i}
                                            }
                                            {\dfrac{e^{i\frac{\theta}{2}} - e^{-i\frac{\theta}{2}}}{2i}}
                                            +
                                            1
                                        }
                        &&  \Remember{Añadimos esto}                                                    \\
                        &= \dfrac{1}{2} \Wrap{
                                            \dfrac{\Sin{(n+\frac{1}{2})\theta}}
                                            {\Sin{\dfrac{\theta}{2}}}
                                            +
                                            1
                                        }
                        &&  \Remember{Recuerda que $\Sin{x} = \dfrac{e^{ix} - e^{-ix}}{2i}$}            \\
                \end{MultiLineEquation*}

            \end{SmallIndentation}





% //////////////////////////////////////////////////////////////////////////////////////////////////////////
% ///////////////////////////////////         NUMEROS COMPLEJOS       //////////////////////////////////////
% //////////////////////////////////////////////////////////////////////////////////////////////////////////
\part{Funciones Complejas}
\clearpage

    % ===============================================================================
    % ===================        DEFINICIONES Y BASES            ====================
    % ===============================================================================
	\chapter{Funciones Complejas}
        \clearpage

        Cualquier función compleja $w = f(z)$ puede ser representada como $f(z) = u(x, y) + iv(x, y)$ donde $x, y \in \Reals$



            

\end{document}